
\documentclass {article}

\setlength {\textwidth} {6in}
\setlength {\textheight} {8in}
\setlength {\topmargin} {0in}
\setlength {\oddsidemargin} {0.25in}
\setlength {\evensidemargin} {0.25in}
\addtolength {\parskip} {1ex}

\newcommand {\caps}[1] {\mbox {\ifnum\fam=0 \scshape\else\uppercase\fi{#1}}}

\newcommand {\tdadvisor} {ETD Administrator}
\newcommand {\oldtdadvisor} {Theses and Dissertations Advisor}
\newcommand {\csd} {Computer Science Department}
\newcommand {\regs} {{\sl Regulations for Thesis
		      and Dissertation Preparation}}
\newcommand {\ucla} {\caps {ucla}}
\newcommand {\uclacsd} {\ucla\ \csd}
\newcommand {\umi} {\caps {umi}}

\title	{ Formatting UCLA Theses and Dissertations \\
	 Using \LaTeX \\
	 and the \texttt{uclathes} Document Style}
\author {Rich Wales}
\date{}
\begin {document}
\maketitle

\newcommand{\MaintainNote}[1]{{\slshape Maintainer's note:  #1}}

\MaintainNote{
  This document was written by Rich Wales for his \texttt{thesis} style.
  I've updated it for the \texttt{uclathes} style for \LaTeX 2e.
  Typos are probably mine.
   ---John Heidemann.}

\MaintainNote{
  This document has been updated to reflect the new electronic process
  adopted by the Graduate Division in 2012 for thesis and dissertation
  submission.
  --- Ryan Rosario.}

This document explains how to format your thesis%
\footnote {In order not to make this document overly verbose,
the term {\em thesis\/} will be used throughout
to indicate either a ``thesis'' (master's degree document)
or a ``dissertation'' (doctor's degree document).
The formatting requirements, in any case,
are virtually identical for both theses and dissertations.
Ph.D.\ students should understand that---%
unless indicated otherwise---%
anything said here about a {\em thesis\/}
applies with equal force to a {\em dissertation\/} as well.}
manuscript using \LaTeX.
It describes a special ``document style'' macro package
which, it is believed, will meet \ucla's requirements
regarding type size, layout, spacing, and margins.

These instructions are not intended to replace
the \ucla\ Graduate Division's publication, \regs.
All graduate students should obtain a copy of this publication,
read it carefully, and check again in advance
of the filing deadline to see if any changes have been made
to the requirements. The manual now comes with a checklist.
This checklist supersedes any outdated, incomplete or vague
 content in the manual.


\section {Thesis and Dissertation Format Requirements}

Theses filed at \ucla\ are required to conform
to certain physical format specifications.
Among the reasons why such formatting issues are important
are the following:

\begin {itemize}

\item
Theses are public, published documents.
A copy of every thesis produced at \ucla\
is filed in one or more University libraries,
and are also available on microfilm
to researchers elsewhere.

Each thesis manuscript is a reflection
of the high standards of the University.
A sloppy manuscript makes the University itself look sloppy,
and so such manuscripts cannot be accepted for filing.

\item
Theses are microfilmed for archival purposes.
Additionally, doctoral dissertations are generally
microfilmed as well by University Microfilms International (\umi).
In order to ensure that the manuscript
will reproduce properly on microfilm,
it is important that the type face is sufficiently large
and that the strokes of the letters are not excessively thin.

\item
In order to guarantee successful binding
of a thesis manuscript in book form---%
as well as to ensure problem-free microfilming---%
the text margins and page number placement
must conform to known standards.

\item
Page numbering must be done in a standardized, consistent fashion,
so as to ensure that errors (such as missed pages)
will not occur in either the binding or the microfilming process.

\end {itemize}

It is crucial that a thesis manuscript
should conform to the University's formatting requirements.
A non-conforming manuscript will \emph{not} be accepted for filing---%
even if the content has been approved
by the student's committee;
even if (in the case of a problem with the signature page)
one or more committee members are not available to sign again;
and even if not enough time remains for the student
to redo the manuscript and file a satisfactory copy by the deadline.
You snooze, you loose.

In order to avoid last-minute disasters,
the \ucla\ \tdadvisor\ strongly urges \emph{all} graduate students
to submit a sample of their thesis manuscript
(including all the preliminary pages)
for review and approval well in advance of the filing deadline.
Think about how long you've been here---can't you really 
  manage to get that dissertation to the online Graduate Division
filing application\footnote{http://grad.ucla.edu/etd}  slightly 
before the last day to file?


\section {Using the \LaTeX\ \texttt{uclathes} Style}

This section describes how to set up
the \LaTeX\ input for your thesis
to use the \verb|uclathes| document style macros.

\subsection {Overall \LaTeX\ Source File Format}

Following are two sample \LaTeX\ input files
which illustrate the proper use of the \verb+uclathes+
document style.

Figure~\ref{fig:toplevel}
shows a ``top-level'' input file,
which itself contains only a minimal skeleton
and includes text from other files
via \verb+\input+ commands.
Figures~\ref{fig:prelim1} and \ref{fig:prelim2}
show the commands for setting up the preliminary pages.

\subsection {The $\backslash${\tt documentclass} Command}

The \verb+\documentclass+ command at the start of the \LaTeX\ input text
should have the following form:

\begin {center}
\verb+\documentclass [+{\sl options\/}\verb+] {uclathes}+
\end {center}

where {\sl options} is one of the following:

\begin {description}

\item [{\tt MS}]
--- M.S.\ thesis.

\item [{\tt PhD}]
--- Ph.D.\ dissertation.

\end {description}

Note that capitalization is important:
\verb+ms+, \verb+PHD+, or \verb+phd+ will {\em not\/} work.

The \verb+uclathes+ document style is basically the same
as the standard \LaTeX\ \verb+report+ style,
as far as the body of the text is concerned.

\subsection {The $\backslash${\tt bibliographystyle} Command}

There is a matching \verb+uclathes+ bibliography style
which should be used in conjunction with
the \verb+uclathes+ document style.
To use the \verb+uclathes+ bibliography style,
use the following \verb+\bibliographystyle+ command
near the end of your \LaTeX\ input
(just before the \verb+\end {document}+ command):

\begin {center} \verb+\bibliographystyle {uclathes}+ \end {center}

The \verb+uclathes+ bibliography style is similar to the
standard Bib\TeX\ \verb+alpha+ style.
One important new feature in the \verb+uclathes+ style is
the addition of a new \verb+annote+ field in a reference,
which can be used to produce an annotated bibliography.

\begin {figure}
{\small
\begin{verbatim}
\documentclass [PhD] {uclathes}
\input {mymacros}                         % personal LaTeX macros
\input {prelim}                           % preliminary page info
\begin {document}
\makeintropages

\chapter{Introduction}
\section{Scope of Research}
Where did we come from, and where will we go? Throughout history, one of humankind's greatest questions is how we came to be, and what might our future hold? Evolutionary biologists have had several competing theories over the development of homospaiens. 
Recent developments in genetic sequencing technology has allowed us to get a more accurate picture of homosapiens and how we interacted with our closest homonid relatives. In 2010 the first Neanderthal genome was sequenced at UCSC. Following this, evidence was found supporting introgression (movement of DNA from one species to another) from Neanderthals and other ancient hominid species into homosapiens (cite). Knowing these, leads us to other relevant questions in how these homonid species impacted our modern human biology. 1 How has neanderthal introgression impacting genetic and phenotypic variation in modern humans. 2 Do these suggest any functional relevance. 3. Was this genomic material harmful or beneficial? 

The relationship of modern humans and our archaic hominid ancestors, such as Neanderthals, has been debated for quite some time. Historically, there were two theories of evolution and migration of modern humans, the multiregional and out-of-Africa models, with evolutionary research seeming to favor the latter [30].  Around the time that modern humans left Africa for Eurasia, archeological evidence shows that other archaic hominids also inhabited these regions, and may have come into contact with modern humans [31]. In recent years, advances in genomic technologies allowed for extraction and analysis of DNA from several of these ancient ancestors including Neanderthals [1, 2], illuminating that fact that admixture occurred between these two species. Further analysis revealed that all present-day non-African populations inherit 1-4\% of their genetic ancestry from a population related to the Neanderthals, and that Neanderthals had lower genetic diversity than any modern human population [1, 2].  Due to this high divergence between the two species, this introgression event introduced many novel mutations into the non-African population. Around the time of this introgression event, archaeological records suggest that modern humans were experiencing behavioral modernity, or cognitive traits such as abstract thinking, which distinguish humans from closely related species.
Systematically studying these mutations has the potential to provide clues about the biological differences between Neanderthals and modern humans, as well as the selective forces that have acted on our genomes in the approximately 50,000 years since Neanderthal introgression occurred. The fact that the period of time since Neanderthal introgression coincides with the period of behavioral modernity evident in the archaeological record [3] suggests that studying the evolution of Neanderthal-derived mutations in modern humans over this period, will give us insight into the nature of natural selection during this critical period of our species’ evolution. 
Analysis of how these Neanderthal segments are distributed in the non-African genome indicates that Neanderthal variants underwent various types of selective pressures[4, 5]. Genomic regions of reduced Neanderthal ancestry are enriched in genes and imply a negative selection of Neanderthal genetic material. One such region is the fivefold reduction of Neanderthal ancestry in the X chromosome, a region known to harbor many male hybrid sterility genes, suggesting Neanderthal alleles caused decreased fertility in males. This is consistent with the hypothesis that the bulk of Neanderthal variants were deleterious in the modern human genetic background [4-7]. 
On the other hand, the frequency of Neanderthal haplotypes is substantially elevated in a small number of genomic locations suggesting evidence for archaic adaptive introgression [1, 5, 8, 9]. Analyses of these genomic locations have suggested that Neanderthal variants could have had an important impact on immune-related as well as skin and hair-related traits, However, the effects of these Neanderthal variants on phenotypes, and selections is still not understood.
In principle a powerful approach to assessing the biological impact of Neanderthal interbreeding is to study Neanderthal-derived mutations in very large cohorts of individuals measured for diverse phenotypes. A recent study employed such an approach to analyze electronic medical records and genotypes in about 28,000 individuals to show that Neanderthal variants modulate risk for disease traits such as major depression, blood-clotting disorders and tobacco use [10]. A difficulty with this approach is that variants introgressed from Neanderthals are rare on average (due to the low proportion of Neanderthal ancestry in present-day genomes) and the genotypes for most rare variants cannot be reliably inferred with the arrays typically used in genetic association studies. Another study analyzed about 112,000 individuals from the interim release of the UK Biobank and identified Neanderthal variants that are individually associated with skin tone, hair color, height, sleeping patterns, mood, and smoking [11]. However, beyond identifying the associations of individual Neanderthal variants, the systematic impact of these variants on a broad spectrum of phenotypes remains to be rigorously assessed. 





"With the addition
of aDNA data, our current atlas of genetic variation
is not limited to a snapshot of the diversity found in
present-day populations across the world. Instead, it is
continuously enriched with temporal information tracking changes in the genetic ancestries of human. aDNA has led to the discovery of new branches within
the human family tree, including that of the Denisovans,
who are close relatives of Neanderthals14–16. As a result,
the genomic consequences of population decline17–19
and the underlying environmental20–22 and/or anthropogenic23,24 drivers of extinctions have been revealed and
clarified. A"                         % Chapter 1 of dissertation
\chapter{Impact of Neanderthal DNA on Depression}

\section{Introduction}
Genetic Relationship with Archaic Hominin Individuals
Past studies of Neandertal genomes have shown that the East Asians have inherited ∼20\% more Neandertal ancestry than Europeans and that this excess ancestry may reflect a second pulse of admixture in East Asians or a dilution of Neandertal admixture in Europeans (Prufer et al. 2014; Sankararaman et al. 2014, 2016; Vernot and Akey 2014, 2015; Kim and Lohmueller 2015). We largely recapitulated the relationship of a number of Neandertal samples and Denisovan to the Han Chinese as previously reported (supplementary fig. S9, Supplementary Material online). We observed subtle differences in allele-sharing pattern and estimated Neandertal ancestry (∼1.8–2\%) across China, though the difference is not significant after correcting for multiple testing (supplementary table S8, Supplementary Material online).
\section{Methods}
Previous analyses of the locations of Neandertal segments within the genomes of non-African individuals indicated that some of the Neandertal variants were adaptively beneficial while the bulk of Neanderthal variants were deleterious in the modern human genetic background (Harris and Nielsen 2016; Juric et al. 2016). Specifically, a recent examination of Neandertal-informative markers (NIMs) among large cohort of Europeans showed that these markers explained some proportions of the phenotypic risk of a number of diseases in the electronic health record (Simonti et al. 2016), including MDD. We sought to replicate this finding in East Asians as our data set was originally ascertained as a case–control study of MDD in Han Chinese women (Cai et al. 2015).

We extracted 75,539 SNPs that were previously identified to tag Neandertal haplotypes in East Asian individuals in the 1KG project (Sankararaman et al. 2014), and assessed the contribution of these NIMs to depression in our cohort consisting 5,224 cases of MDD and 5,218 controls. The allele frequencies of these NIMs are highly correlated $(r=0.951)$ between our cohort and 1KG, suggesting that the NIMs are not overt outliers from the rest of the variants in our data set in terms of data quality. We tested the association between the NIMs and depression by performing a logistic regression of depression, controlling for age and the first ten PCs, for MDD and Melancholia. Using the current sample size and sequence data, we found no association surviving the Bonferroni correction (supplementary fig. S10, Supplementary Material online) and the QQ plots did not reveal any systematic inflation nor significant enrichment among top associated SNPs (data were not shown).

We also calculated the proportion of phenotypic variance explained by these NIMs using GCTA (Yang et al. 2011) for MDD. We used a prevalence of 7.5\% to transform the heritability to the liability scale. We found that the variance explained by the NIMs is ∼1\%, which is different from that reported in Simonti et al. (∼2\%) and is not significantly different from 0 $(P=0.12)$. Repeating the analysis with NIMs with MAF >0.01 as well as with no covariates did not qualitatively alter the results (supplementary table S9, Supplementary Material online). Finally, we found that the heritability explained by NIMs is not significantly different from that of a background set of SNPs chosen at random to match the NIMs by derived allele frequency decile and by Linkage Disequilibrium (LD) scores $(P>0.4)$. Our analysis may be underpowered given the smaller sample size and low coverage, but the results could suggest that the impact of Neandertal ancestry on MDD differs between European and Han Chinese. Future investigation in larger cohorts will be informative.
%***********PBC***************
\section{PBC}
                         % Chapter 2
\chapter{Impact of Neanderthal DNA on UK Biobank Phenotypes}
\section{Introduction}
Genomic analyses have revealed that present-day non-African human populations inherit 1-4\% of their genetic ancestry from introgression with Neanderthals (Green 2010, Prufer 2014). This introgression event introduced uniquely Neanderthal variants into the ancestral out-of-Africa human gene pool, which may have helped this bottleneck population survive the new environments they encountered (Mendez 2012, Abi-Rached 2011, Sankararaman 2014, Vernot 2014, Racimo 2015, Gittelman 2016). On the other hand, the bulk of Neanderthal variants appear to have been deleterious in the modern human genetic background leading to a reduction in Neanderthal ancestry in conserved genomic regions (Sankararaman 2014, Vernot 2014, Harris 2016, Juric 2016, Petr 2019). Systematically studying these variants can provide insights into the biological differences between Neanderthals and modern humans and the evolution of human phenotypes in the 50,000 years since introgression. 

In principle, studying Neanderthal-derived mutations in large cohorts of individuals measured for diverse phenotypes can help understand the biological impact of Neanderthal introgression. Previously, Dannemann and Kelso (Dannemann 2017) showed that some Neanderthal introgressed variants are significantly associated with traits such as skin tone, hair color, and height based on Genome-Wide Association Studies (GWAS) in British samples. However, using data from Iceland, Skov et al. (Skov 2020) found that most of the significantly associated Neanderthal introgressed SNPs are in the proximity of strongly associated non-archaic variants. They suggested that these associations at Neanderthal introgressed SNPs were driven by the associations at linked non-archaic variants, indicating a limited contribution to modern human phenotypes from Neanderthal introgression. In contrast to these attempts to associate individual introgressed variants with a trait, studies have attempted to measure the aggregate contribution of introgressed Neanderthal SNPs to trait variation (Simonti 2016, McArthur 2021). A recent study by McArthur and colleagues (McArthur 2021) estimated the proportion of heritable variation that can be attributed to introgressed variants though their approach is restricted to common variants (minor allele frequency $> 5\%$)  that represent a minority of introgressed variants. Despite these attempts, assessing the contribution of introgressed Neanderthal variants towards specific phenotypes remains challenging. The first challenge is that variants introgressed from Neanderthals are rare on average (due to the low proportion of Neanderthal ancestry in present-day genomes). The second challenge arises from the unique evolutionary history of introgressed Neanderthal variants resulting in distinct population genetic properties at these variants which can, in turn, confound attempts to characterize their effects. As a result, attempts to characterize the systematic impact of introgressed variants on complex phenotypes need to be rigorously assessed.

To enable analyses of genome-wide introgressed Neanderthal variants in large sample sizes, we selected and added Single Nucleotide Polymorphism (SNPs) that tag introgressed Neanderthal variants to the UKBiobank Axiom Array that was used to genotype the great majority of the approximately 500,000 individuals in the UK Biobank (UKBB) (Bycroft 2018) . We used a previously compiled map of Neanderthal haplotypes in the 1000 Genomes European populations (Sankararaman 2014) to identify introgressed SNPs that tag these haplotypes. After removing SNPs that are well-tagged by those previously present on the UK Biobank (UKBB) array, we used a greedy algorithm to select 6,027 SNPs that tag the remaining set of introgressed SNPs at $r^2>0.8$ which were then added to the UKBB genotyping array to better tag Neanderthal ancestry. These SNPs allow variants of Neanderthal ancestry to be confidently imputed and allow us to identify a list of 235,592 mutations that are likely to be Neanderthal-derived (termed Neanderthal Informative Mutations or NIMs) out of a total of  7,774,235 QC-ed SNPs in UKBB (see Methods; Note S1). 

The goals of our study are threefold: 1) to estimate the contribution of NIMs to phenotypic variation in modern humans, 2) to test the null hypothesis that a NIM has the same contribution to phenotypic variation as a non-introgressed modern human SNP, and 3) to pinpoint regions of the genome at which NIMs are highly likely to modulate phenotypic variation. We develop rigorous methodology for each of these goals which we validate in simulations. We then applied these methods to 96 distinct phenotypes measured in about 300,000 unrelated white British individuals in UKBB.
\section{Results}
\subsection{The contribution of Neanderthal introgressed variants to trait heritability}
To understand the contribution of Neanderthal introgressed variants to trait variation, we aim to estimate the proportion of phenotypic variance attributed to NIMs (NIM heritability) and to test the null hypothesis that per-NIM heritability is the same as the heritability of a non-introgressed modern human (MH) SNP. We first annotated each of the 7,774,235 QC-ed SNPs in UKBB as either a NIM or a MH SNP (see Methods). NIMs include SNPs created by mutations which likely originated in the Neanderthal lineage after the human-Neanderthal split. SNPs that are not defined as NIMs are annotated as MH SNPs which likely originated in the modern human lineage or the human-Neanderthal common ancestor. 

To estimate NIM heritability, we used a recently proposed method (RHE-mc) that can partition the heritability of a phenotype measured in large samples across various genomic annotations (Pazokitoroudi 2020). We applied RHE-mc with genomic annotations that correspond to the ancestry of each SNP (NIM vs MH) to estimate NIM heritability ($h^2_{NIM}$). We also attempted to estimate whether per-NIM heritability is the same as the per-SNP heritability of MH SNPs ($\Delta_{h^2}$). A positive (negative) value of $\Delta_{h^2}$ indicates that, on average, a NIM makes a larger (smaller) contribution to phenotypic variation relative to a MH SNP.

To assess the accuracy of this approach, we performed simulations where NIMs are neither enriched nor depleted in heritability (true $\Delta_{h^2}=0$). Following previous studies of the genetic architecture of complex traits (Evans 2018, Gazal 2018), we simulated phenotypes (across 291,273 unrelated white British individuals and 7,774,235 SNPs) with different architectures where we varied heritability, polygenicity, and how the effect size at a SNP is coupled to its population genetic properties (the minor allele frequency or MAF at the SNP and the linkage disequilibrium or LD around a SNP). We explored different forms of MAF-LD coupling where BASELINE assumes that SNPs with phenotypic effects are chosen randomly, RARE (COMMON) assumes that rare (common) variants are enriched for phenotypic effects, and HIGH (LOW) assumes that SNPs with high (low) levels of LD (as measured by the LD score (Finucane 2015)) are enriched for phenotypic effects (see Methods). Estimates of $h^2_{NIM}$ and $\Delta_{h^2}$ tend to be miscalibrated (Fig. 1ab). The miscalibration is particularly severe when testing $\Delta_{h^2}$ so that a test of the null hypothesis has a false positive rate of 0.55 across all simulations (at a p-value threshold of 0.05).

To understand these observations, we compared the Minor Allele Frequencies (MAF) and LD scores at NIMs to MH SNPs. We observe that NIMs tend to have lower MAF (Fig.2a) and higher LD scores compared to MH SNPs (Fig. 2b) (the average MAF of NIMs and MH SNPs are 3.9\% and 9.9\%, respectively while their average LD scores are 170.6 and 64.9). Among the QC-ed SNPs, 76.9\% of NIMs have $MAF > 1\%$, and 27.7\% have $MAF > 5\%$, in contrast to 61.6\% and 41.6\% of MH SNPs. Distinct from MH SNPs, the MAF and LD score of NIMs tend not to increase with each other (Fig. 2cd). 

To account for the differences in the MAF and LD scores across NIMs and MH SNPs, we applied RHE-mc with annotations corresponding to the MAF and the LD score at each SNP (in addition to the ancestry annotation that classifies SNPs as NIM vs. MH) to estimate NIM heritability ($h^2_{NIM}$) and to test whether per-NIM heritability is the same as the per-SNP heritability of MH SNPs i.e., $\Delta_{h^2}=0$ (see Methods, Note S4). Our simulations show that RHE-mc with SNPs assigned to annotations that account for both MAF and LD (in addition to the ancestry annotation that classifies SNPs as NIM vs. MH) is accurate both in the estimates of hNIM2 (Fig. 1a) and in testing the null hypothesis that $\Delta_{h^2}=0$ (the false positive rate of a test of $\Delta_{h^2} = 0$  is $0.017$ at a p-value threshold of 0.05; Fig. 1b). On the other hand, not accounting for either MAF or LD leads to poor calibration (Fig. 1; we observe qualitatively similar results when estimating genome-wide SNP heritability; Fig S1).

We then applied RHE-mc with ancestry+MAF+LD annotations to analyze a total of 96 UKBB phenotypes that span 14 broad categories (Data S2). In all our analyses, we include the top five PCs estimated from NIMs (NIM PCs) as covariates in addition to the top twenty genetic PCs estimated from common SNPs, sex, and age (see Methods). The inclusion of NIM PCs is intended to account for stratification at NIMs that may not be adequately corrected by including genotypic PCs estimated from common SNPs (we also report concordant results from our analyses when excluding NIM PCs; Note S3 and Fig. S3-S4).

We first examined NIM heritability to find six phenotypes with significant NIM heritability (Z-score $(\hat{h^2_{NIM}}=0)>3$ ): body fat percentage, trunk fat percentage, whole body fat mass, overall health rating, gamma glutamyltransferase (a measure of liver function), and forced vital capacity (FVC) (Fig. 3ac). Meta-analyzing within nine categories that contain at least four phenotypes, we find that $meta-\hat{h^2_{NIM}}$ is significantly larger than zero for anthropometry, blood biochemistry, bone densitometry, kidney, liver, and lung but not for blood pressure, eye, lipid metabolism ($p < 0.05$ accounting for the number of hypotheses tested).  Meta-analyzing across all phenotypes with low correlation, we obtain overall NIM heritability estimates ($meta -\hat{h^2_{NIM}}$)$=0.1\%$ (one-sided $p=9.59\times10^{-9}$). The estimates of NIM heritability are modest as would be expected from traits that are highly polygenic and given that NIMs account for a small percentage of all SNPs in the genome (see Methods). 

We next tested whether the average heritability at a NIM is larger or smaller compared to a MH SNP ($\hat{\Delta_{h^2}}=0$). We find seventeen phenotypes with significant evidence of depleted NIM heritability that include standing height, body mass index, and HDL cholesterol ($Z-score < -3$; \Fig. 3bd). Five phenotypic categories show significant NIM heritability depletion (anthropometry, blood biochemistry, blood pressure, lipid metabolism, lung) in meta-analysis. Meta-analyzing across phenotypes, we find a significant depletion in NIM heritability ($meta-\hat{\Delta_{h^2}} = -1.4\times10^{-3},\ p= 2.55\times10^{-11}$). On average, we find that heritability at NIMs is reduced by about 57\% relative to a modern human variant with matched MAF and LD characteristics. In contrast to the evidence for depletion in NIM heritability, we find no evidence for traits with elevated NIM heritability across the phenotypes analyzed. Despite the observation that NIMs have been primarily under purifying selection for thousands of generations (Harris and Nielsen 2016, Petr 2019), they still make a substantial contribution to phenotypic variation in present-day humans. 
  
Finally, we investigated the impact of controlling for MAF and LD on our findings in UKBB. Analyses that do not control for MAF and LD tend to broadly correlate with our results that control for both (Pearson’s $r = 0.96, 0.68$, and $0.65$ and $p < 10-12$ among $\hat{h^2}$, $\hat{h^2_{NIM}}$, and $\hat{\Delta_{h^2}}$). However, these analyses underestimate both heritability (Fig. 4a) and NIM heritability (Fig. 4b), resulting in apparent NIM heritability depletion $(Z-score < -3)$ in 83 of the 96 phenotypes (Fig. 4c). While yielding qualitatively similar conclusions about the depletion in heritability at NIMs relative to MH SNPs, prior knowledge that per SNP heritability of complex traits can be MAF and LD dependent (Evans 2018) coupled with our extensive simulations lead us to conclude that controlling for MAF and LD lead to more accurate results. 

\subsection{Identifying genomic regions at which introgressed variants influence phenotypes}
Having documented an overall contribution of NIMs to phenotypic variation, we focus on identifying individual introgressed variants that modulate variation in complex traits. We first tested individual NIMs for association with each of 96 phenotypes (controlling for age, sex, twenty genetic PCs (estimated from common SNPs), and five NIM PCs (that account for potential stratification that is unique to NIMs). We obtained a total of 13,075 significant NIM-phenotype associations in 64 phenotypes with 8,018 unique NIMs ($p < 10e-10$ that accounts for the number of SNPs and phenotypes tested)  from which we obtain 348 significant NIM-phenotype associations with 294 unique NIMs after clumping associated NIMs by LD (see Methods).
 
A limitation of the association testing approach is the possibility that a NIM might appear to be associated with a phenotype simply due to being in LD with a non-introgressed variant (Skov 2020). We formally assessed this approach in simulations of phenotypes with diverse genetic architectures described previously where the identities of causal SNPs are known. A NIM that was found to be associated with a phenotype ($p < 10^-10$) was declared a true positive if the 200 kb region surrounding the associated NIM contains any NIM with a non-zero effect on the phenotype and a false positive otherwise. Averaging across all genetic architectures, the False Discovery Proportion (FDP; the fraction of false positives among the significant NIMs) of the association testing approach is around 30\% (Fig. 5b). Hence, finding NIMs that are significantly associated with a phenotype does not confidently localize regions at which introgressed variants affect phenotypes.

To improve our ability to identify NIMs that truly modulate phenotype, we designed a customized pipeline that combines association testing with a fine-mapping approach that integrates over the uncertainty in the identities of causal SNPs to identify sets of NIMs that plausibly explain the association signals at a region (Fig. 5a). Our pipeline starts with a subset of significantly associated NIMs that are relatively independent $(p < 10^-10)$ followed by the application of a statistical fine-mapping method (SuSiE) within the 200kb window around each NIM signal (Wang 2020) and additional post-processing to obtain a set of NIMs that have an increased probability of being causal for a trait. We term the NIMs within this set credible NIMs while the shortest region that contains all credible NIMs in a credible set is termed the credible NIM region (see Methods; Fig. 5a). 

We employed the same simulations as previously described to evaluate our fine-mapping approach. The fine mapping approach yields a reduction in the FDP relative to association mapping (FDP of 15.6\% on average; Fig. 5b) while attributing the causal effect to a few dozen NIMs within the credible NIM set (mean: 79, median: 54 NIMs across all simulations). Applying our pipeline to the set of 96 UKBB phenotypes, we identified a total of 112 credible NIM regions containing 4,303 unique credible NIMs across 47 phenotypes (Fig. 6a). The median length of credible NIM regions, 65.7kb $(95\% CI: [4.41kb, 469.3 kb])$ is close to the expected length of Neanderthal introgressed segments (Skov 2020) suggesting that the resolution of our approach is that of an introgressed LD block (Fig. 5c). While fine mapping generally attributes the causal signal to a subset of the tested NIMs (mean: 55.8, median: 37 NIMs across phenotypes), the degree of this reduction varies across regions likely reflecting differences in the LD among NIMs (Fig. 5d). We do not detect any credible NIM in 49 out of 96 phenotypes potentially due to the limited power of our procedure that aims to control the FDR (Fig. 5e). The sensitivity of our method is affected by both total heritability (Fig. 5f, Pearson’s $r = 0.49 , p = 3.3e-7)$  and NIM heritability (Fig. 5g, Pearson’s $r = 0.36, p = 3.3e-4)$. A linear model that uses both total heritability and NIM heritability to predict the number of credible sets yields $r2 = 0.29, p = 1.310-5 and 0.015$, respectively), while linear models with only total heritability or only NIM heritability result in statistically lower $r2 (0.24 and 0.13$, respectively).
\subsection{Examination of the functional impact of credible NIMs}
We annotated all 4,303 unique credible NIMs using SnpEff (Cingolani 2012) to identify a total of 26 NIMs with high (e.g., start codon loss, stop codon gain) or moderate impact (nonsynonymous variants) on genes (Fig. 6b, SI Data S7). We identified two credible NIMs, rs9427397 $(1:161,476,204 C>T)$ and rs60542959 $(12:56,660,905 G>T)$, that have a high impact on protein sequences. The 1:161,476,204 C>T mutation, a NIM that is associated with increased gamma glutamyltransferase and aspartate aminotransferase (enzymes associated with liver function) and decreased total protein levels in blood, introduces a premature stop codon in the FCGR2A gene (Fig. S7). FCGR2A codes for a receptor in many immune cells, such as macrophages and neutrophils, and is involved in the process of phagocytosis and clearing of immune complexes. This NIM is in a region that contains SNPs shown in several GWAS  linked to rheumatoid arthritis (Okada 2014, Laufer 2019). The other high impact mutation, $12:56,660,905 G>T (rs60542959)$, results in the loss of the start codon in COQ10A, and this SNP is a credible NIM for both mean platelet volume and standing height (Fig. 6c) . COQ10 genes (A and B) are important in respiratory chain reactions. Deficiencies of CoQ10 (MIM 607426) have been associated with encephalomyopathy, infantile multisystemic disease; cerebellar ataxia, and pure myopathy (Quinzii 2009). The start codon in COQ10A is conserved among mammals with its loss having a potentially significant effect on COQ10A expression in immune cells (Kubota 2020).

In addition, we detect 24 credible NIMs that function as missense mutations in 19 genes. Seven out of the 19 genes are known to have immune related functions (FCGR2A, PCDHG (A8, A9, B7, C4), STAT2, and IKZF3).  The NIM in STAT2 $(rs2066807, 12:56,740,682 C>G)$ was the first adaptive introgression locus to be identified (Mendez 2012). The STAT2 introgressed variant segregates at 0.066 frequency in the UKBB white British and leads to an I594M amino acid change in the corresponding protein. STAT2 gene and COQ10A are neighboring genes thereby providing an example of an introgressed region that potentially impacts function at multiple genes (Fig. 6c). 

At least seven of the 12 genes not known to be immune related have other important functions documented in the literature, such as DNA replication/damage (FANCA, CCDC8), transition in meiosis (FBXO34), detoxification/metabolism (AKR1C4), and neurological/developmental (ZNF778, ANKRD11, TBC1D32) functions. $rs17134592 (10:5260682 C>G)$ is a non-synonymous mutation in AKR1C4, a gene that is involved in the metabolism of ketone-containing steroids in the liver. The NIM is associated with increased serum bilirubin levels $(p = 3e-11)$  (Fig. S7a) while also being associated with increased levels of alkaline phosphatase, insulin-like growth factor 1 (IGF1) and decreased apolipoprotein A, sex hormone binding globulin (SHBG) and triglyceride levels.  rs17134592 has been identified to be a splicing QTL that is active in the liver and testis in the GTeX data (Fig. S7b). This NIM alters Leucine to Valine (L311V) which, in combination with the tightly-linked non-synonymous variant rs3829125 (S145C) in the same gene, have been shown to confer a three-to-five-fold reduction in catalytic activity of the corresponding enzyme (3-alpha hydroxysteroid dehydrogenase) in human liver (Kume 1999). Interestingly, the single amino acid change S145C did not significantly alter enzyme activity suggesting the importance of the amino acid residue at position 311 for the substrate binding of the enzyme.
\section{Methods}
Identification and design of SNPs that tag Neanderthal ancestry on the UK Biobank Axiom array 
We chose a subset of SNPs to add to the UKBiobank Axiom array that would tag introgressed Neanderthal alleles segregating in present-day European populations.
 
We began with a list of 95,462 SNPs that are likely to be Neanderthal-derived from Sankararaman et al. 2014. These SNPs were identified to tag confidently inferred Neanderthal haplotypes in the European individuals identified in the 1000 Genomes Phase 1 data (Note S1).

We winnowed down this list to 43,026 SNPs after removing ones already tagged at r2>0.8 by SNPs on the UKBiLEVE array. We then designed a greedy algorithm to capture the remaining untagged SNPs that could still be accommodated on the array (we determined the number of oligonucleotide features that would be needed to genotype each SNP as well as the total number of features available on the array through discussions with UKBiobank Axiom array design team).

Specifically, we computed LD between all pairs of Neanderthal-derived SNPs and then iteratively picked SNPs with the highest score to add to the array where the score was computed as:
$$ScoreSNP j = i=1n [r2>0.80 (i,j) ][Derived frequencySNP i ]Features required to genotype SNP j$$
Here, r2>0.80 (i,j) is an indicator variable that is 1 if the squared correlation coefficient between SNPs i and j is >0.80 and zero otherwise. Thus, SNP j is scored higher if it tags other untagged SNPs on the array. The other two terms upweight SNPs that tag other Neanderthal-derived SNPs with high derived allele frequency in Europeans and downweight SNPs by the number of oligonucleotide features required to genotype the SNP.

We iteratively chose SNPs until we obtained 6,027 SNPs (requiring 16,674 features) that fully tagged the remaining set of Neanderthal-derived SNPs. These 6,027 SNPs were then added to the UKBiobank Axiom array.

UK Biobank (UKBB) genotype QC 
We restricted all our analyses to a set of high-quality imputed SNPs (with a hard call threshold of 0.2 and an info score greater than or equal to 0.8), which, among the 291,273 imputed genotypes of UKBB unrelated white British individuals, 1) have MAF higher than 0.001, 2) are under Hardy-Weinberg equilibrium (p > 10-7), and 3) are confidently imputed in more than 99\% of the genomes. Additionally, we excluded SNPs in the MHC region, resulting in a total of 7,774,235 SNP which we refer to as QC-ed SNPs. 

Identification of Neanderthal Informative Mutations
We intersected the 95,462 Neanderthal-derived SNPs identified in the 1000 Genomes European individuals with UKBB QC-ed SNPs, resulting in 70,374 mutations that we term confident Neanderthal Informative Mutations (NIM). SNPs in high Linkage Disequilibrium (LD) with this set are likely introduced through Neanderthal introgression. We expanded this set by including all QC-ed SNPs, which 1) have an r2 of 0.99 or higher with any confident NIM, and 2) are located in the proximal neighborhood of any confident NIM (within 200kb). We term this set of SNPs as expanded NIMs. On average, 80.58\% of expanded NIMs match the corresponding Altai Neanderthal allele, in contrast to 2.18\% of the remaining SNPs, suggesting that these SNPs are also highly informative about Neanderthal ancestry. This treatment expands the number of NIMs in the UKBB QC-ed SNPs from 70,374 (confident NIMs) to 235,592 (expanded NIMs). We primarily use this more inclusive set of SNPs in our analyses, and refer to them as NIMs in the main results. SNPs that were not part of the expanded NIMs are termed modern human (MH) SNPs.

Annotating QC-ed SNPs by MAF and LD 
In addition to ancestry (Neanderthal vs MH), we annotate each QC-ed SNP by its minor allele frequency (MAF) and LD. We define five MAF-based annotations by dividing all QC-ed SNPs into five equal-sized bins by their MAFs. We similarly define five LD-based annotations by dividing all QC-ed SNPs into five equal-sized bins based on their LD-score computed from 291,273 imputed unrelated white British genotypes. In-sample LD-score is computed on QC-ed genotypes using GCTA (https://cnsgenomics.com/software/gcta/#Overview)  with flags “--ld-score --ld-wind 10000”.

After each QC-ed SNP is annotated with three properties -- ancestry (NIMs vs MH), MAF, and LD, we use them to construct three additional sets of annotations: ancestry + MAF, ancestry + LD, and ancestry + MAF + LD annotations, by intersecting MAF annotation with ancestry annotation, LD annotation with ancestry annotation, and all three annotations, respectively. For example, for ancestry + MAF annotation, we intersect the previously defined MAF annotation with the ancestry annotation and divide SNPs into ten non-overlapping bins -- from low to high MAF with Neanderthal ancestry (five bins) and from low to high MAF with modern human ancestry (five bins). Similarly, when SNPs are annotated with LD + ancestry, we have five LD bins with Neanderthal ancestry corresponding to five LD groups with modern human ancestry. 

Because NIMs tend to have low MAF and high LD-score (Fig. 2), the sizes of the annotation bins are highly uneven. To enable reliable downstream heritability analyses, we remove the annotation bins in their entirety if they include fewer than 30 SNPs. Such exceptions only occur when SNPs are annotated based on all three annotations, i.e., ancestry + MAF + LD. 

Whole-genome simulations
We simulated phenotypes based on QC-ed UKBB genotypes with the same sample size (291,273) and number of SNPs (7,774,235). In each simulation, either 10,000 variants (mimicking moderate polygenicity) or 100,000 (mimicking high polygenicity) are sampled from the QC-ed SNPs to have causal phenotypic effects while the rest of the variants have zero effect. Causal effects and phenotypes are simulated with GCTA assuming either a high SNP heritability of 0.5 or a moderate SNP heritability of 0.2.

With the simulated causal NIM variants, true NIM heritability hNIM2can be computed as
hNIM2= iNIM,i2/Var(y) 
where phenotypes y are simulated based on a set of standardized genotype data with a simple additive genetic model
yj=iwiji+j
and 
wij=(xij-2pi)/2pi(1 - pi)
with xij being the number of reference alleles for the ith causal variant of the jth individual and pi being the frequency of the ith causal variant, iis the allelic effect of the ith causal variant and j is the residual effect generated from a normal distribution with mean 0 and variance Var(iwiji)/(1/$h^2-1). 

Following previous work (Evans 2018), we chose causal variants according to five different MAF and LD-dependent genetic architecture : 1) BASELINE: baseline architecture, where SNPs are randomly selected to be causal variants, 2) COMMON: common SNPs are enriched for phenotypic effects so that SNPs with MAF > 0.05 contribute 90\% of causal variants while rare SNPs contribute 10\%, 3) RARE: rare variants are enriched for phenotypic effects such that SNPs with MAF <= 0.05 contribute to 90\% of causal variants while the rest contribute 10\%, 4) LOW: low LD SNPs are enriched for phenotypic effects, realized as SNPs whose LD-score <= 10 contribute 90\% of causal variants, and the rest contribute 10\%, and 5) HIGH: high LD SNPs are enriched for phenotypic effects, such that SNPs with LD-score > 10 contribute 90\% causal variants while the rest contribute 10\%. We simulated three replicates, for each genetic architecture with two different values of SNP heritability (0.2 and 0.5) and two different levels of polygenicity (10,000 and 100,000 causal variants). Thus, we simulated a total of 60 genetic architectures.

Estimating NIM heritability with RHE-mc
We are interested in estimating the proportion of phenotypic variance attributed to NIMs (true NIM heritability hNIM2) and evaluating if the heritability at a NIM (per-NIM heritability) is larger or smaller than that of a background MH SNP. To this end, we used a variance components model that partitions phenotypic variance across genomic annotations that include ancestry (NIM vs MH) as one of the input annotations.

We use RHE-mc, a method that can partition genetic variance across large sample sizes, to estimate NIM heritability (Pazokitoroudi 2020). For each phenotype, we run RHE-mc, in turn, with four types of input annotations: ancestry alone, ancestry + MAF, ancestry + LD, and ancestry + MAF + LD as described above. The ancestry+ MAF, ancestry + LD, and ancestry + MAF + LD annotations are intended to account for the differences in the MAF and LD properties of NIMs compared to MH SNPs.

To estimate NIM heritability, $h^2_{NIM}$ , we combine the heritability of each bin corresponding to Neanderthal ancestry:
$h^2_{NIM}$ = i$h^2_{NIM}$, i
and the heritability estimates for any bins with modern human ancestry are used to compute the total heritability from MH. Thus, when we estimate NIM heritability from RHE-mc run with ancestry + MAF annotations, we add the heritability estimates from five bins of low to high MAF NIMs.
 
To compare the average heritability at a NIM to the heritability of a background MH SNP that is chosen to match the NIM in terms of MAF and LD profiles, we compute the following statistic:
$h^2=h^2_{NIM}-h^2_{MH}$ 
where $h^2_{MH}$ = i MNIM,iMMH,i $h^2_{MH}$, iis the heritability of the background set matched for the MAF and LD profile of the set of NIMs. Here MMH,i denotes the number of MH SNPs in bin i  (defined according to MAF and/or LD of the MH SNPs) while MNIM,i denotes the number of NIMs in the corresponding bin. A more detailed justification of this statistic is provided in Note S4.

The standard errors (s.e.) of these statistics are computed using 100 jackknife blocks using an extension of RHE-mc that takes into account the covariance among different annotations. This new version of the RHE-mc is now available at https://github.com/alipazokit/RHEmc-coeff. 

NIM heritability and META-analysis using UKBB phenotypes
We applied RHE-mc to a total of 96 UKBBphenotypes. These phenotypes fall into 14 broader phenotypic categories (Data S1): anthropometry, autoimmune disorders, blood biochemistry, blood pressure, bone densitometry, environmental factors, eye, general medical information, glucose metabolism, kidney, lipid metabolism, liver, lung, and skin and hair. For each phenotype, we use RHE-mc to estimate the NIM heritability $h^2_{NIM}$ and the difference between per-NIM heritability and the per-SNP heritability of MH SNPs h2 while controlling for age, sex, the first 20 genetic Principal Components (PCs) estimated from common SNPs, and the first five PCs estimated from NIMs (NIM PCs). The five NIM PCs are computed using all NIMs in unrelated white British samples with ProPCA (Agrawal 2020). 


To improve power to detect patterns that are shared across groups of phenotypes, we combined analyses across groups of phenotypes and across all phenotypes analyzed. We performed random effect meta-analysis on each phenotypic category containing at least four phenotypes. We assume that the phenotypes within each category i have their $h^2_{NIM}$ drawn from the same distribution so that we can estimate the mean (meta-hNIM2) and variance of distribution i, based on the sampled $h^2_{NIM}$and the s.e.($h^2_{NIM}$). From there, we computed the meta analysis Z-score to test if the meta-hNIM2is equal to zero. Similarly, we assume the phenotypes within each category i have their h2drawn from the same distribution, and compute the Z-score to test if the meta-h2is equal to zero. In addition to the meta-analysis within the phenotypic category, we also performed meta-analysis across all phenotypes where we used a subset of 32 phenotypes that were chosen to have low correlation (Pearson’s r2   0.25 ). 

Identifying individual NIMs associated with phenotype
To identify individual NIMs associated with a phenotype, we fit a linear regression model using plink 2.0 --glm and included covariates controlling for age, sex and the first 20 genotypic PCs, and first five NIM PCs. We used  a stringent p-value threshold of 10-10 to correct for the number of NIMs and phenotypes tested. For each phenotype, we clumped all significant NIMs that lie within 250 kb and with an LD threshold ( r2) of 0.5 using a significance threshold for the index SNP of 10-10.


Identifying NIMs that modulate phenotype
To assess our ability to identify introgressed variants that truly modulate a phenotype, we first tested each NIM for association with the simulated phenotype. A challenge with such an approach is the possibility that a NIM can be found to be associated with a phenotype due to being in LD with a non-introgressed variant. To exclude settings where the association signal at a NIM might be driven by LD with a non-introgressed variant, we applied a Bayesian statistical fine-mapping method (SuSiE, https://stephenslab.github.io/susie-paper/index.html) that analyzes both NIM and MH SNPs in the region surrounding an associated NIM to output a set of SNPs that can explain the association signal at the region. Furthermore, we processed these credible sets to obtain a set of credible NIMs. 

We performed simulations to test the accuracy of such an approach in identifying truly causal NIMs. In particular, we first ran an association test with plink (https://www.cog-genomics.org/plink/) to identify significant NIMs (p-value < 10-10). We then LD-pruned significant NIMs to get a subset of NIMs which are approximately uncorrelated with each other (using the plink flag “--indep-pairwise 100kb 1 0.99”). For each LD-pruned significant NIM, we considered all the QC-ed SNPs in its 200kb neighborhood as input to fine mapping. We ran SuSiE with $⍴ = 0.95$ and $L = 10$, such that it returns credible sets that have at least 0.95 probability to contain one causal variant and outputs at most ten credible sets for each tested region. If there are more than one credible set for a tested region, we merge them into one set. We then removed the credible sets which have 50\% or more MH SNPs in their credible set. The remaining credible sets all have majority NIMs (i.e. positive results), and they are further merged together with other such regions it overlaps with, resulting in distinct regions with evidence of NIM causal effects. We termed the set of all resulting NIMs as the credible NIM set and all NIMs that lie in the credible set as credible NIMs. The region containing the credible NIM set is termed credible NIM region. If there is at least one true causal NIM within the set of credible NIMs, this credible NIM region is counted as a True Positive (TP). If there is no causal NIM in the credible NIMs, this credible NIM region is counted as a False Positive (FP). 
 
We adopted the same approach when analyzing UKBB phenotypes while incorporating covariates. Because the SuSiE package does not directly incorporate covariates, we used regression residuals from linear regression between each UKBB phenotype and UKBB covariates (age, sex, 20 regular PCs, 5 NIM PCs), as the input phenotype to SuSiE.   

Annotating NIMs
We annotated all unique credible NIMs using SnpEff (Cingolani 2012) which uses Sequence Ontology (http://www.sequenceontology.org/) to assign standardized terminology for assessing sequence change and impact. We primarily focused on examining the high (e.g., start codon loss, stop codon gain) and moderate impact SNPs (nonsynonymous variants) which are coding variants that alter protein sequences.

\section{Discussion}
Our analysis demonstrates the complex influence of Neanderthal introgression on complex human phenotypes. The assessment of the overall contribution of introgressed Neanderthal alleles to phenotypic variation indicates a pattern where, taken as a group, these alleles tend to be depleted in their impact on phenotypic variation (with about a third of the studied phenotypes showing evidence of depletion). This pattern is consistent with these alleles having entered the modern human population roughly 50,000 years ago and being subject to purifying selection. Selection to purify deleterious introgressed variants, coupled with stabilizing selection on human complex traits, could result in introgressed heritability depletion such that the remaining introgressed variants in present-day humans tend to have smaller phenotypic effects compared to other modern human variants. 

Nevertheless, we document a modest but significant contribution of introgressed alleles to variation in a number of phenotypes. In contrast to the previous heritability analyses by McArthur et al. (McArthur 2020), we did not find any NIM heritability enrichment in the 96 phenotypes. This discrepancy could be due to the different methods and NIMs used in the two studies. McArthur et al. estimate the heritability associated with common NIMs (NIMs with $MAF > 5\%$) using stratified LD Score Regression (S-LDSR) with LD scores computed from 1KG (see Note S2). Because more than 70\% of NIMs have $MAF < 5\%$, this approach may not extrapolate to understand the heritability from all NIMs. An additional potential concern with analyses of NIMs is the possibility of confounding due to population structure among these introgressed variants. Typical approaches to account for population stratification based on the inclusion of principal components (PCs) may not be adequate as these PCs are computed from common SNPs on the UK Biobank genotyping array and may not account for stratification at the NIMs that tend to be rare on average (Mathieson 2012). Since our analyses work directly on individual genotype data, we are able to control for stratification specific to NIMs by including PCs estimated from NIMs in addition to PCs estimated from common SNPs. Our analyses are broadly consistent when including NIM PCs than without (see Note S3). 

Beyond characterizing aggregate effects of NIMs, we also attempted to identify individual NIMs that modulate phenotypic variation. A challenge in identifying such variants comes from the fact that NIMs tend to have lower MAF and higher LD compared to MH SNPs. Lower MAF tends to limit the power to detect a genetic effect while higher LD makes it harder to identify the causal variant. These challenges led us to design a fine mapping strategy for prioritizing causal NIMs that enables the identification of sets of NIMs that can credibly exert influence on specific phenotypes. Using this approach, we identified credible NIMs in a number of functionally important genes, including a premature stop codon in the FCGR2A gene, and a start codon loss in COQ10A. In addition, mutations in STAT2 are found to be highly pleiotropic. As many of the genes are relevant to immune, metabolic, and developmental disorders, with functions relevant to the transition to new environments, the credible NIMs reported in our study offer a starting point for detailed investigation of the biological effects of introgressed variants. Greenbaum et al. hypothesized that introgression-based transmission of alleles related to the immune system could have helped human out-of-Africa expansion in the presence of new pathogens (Greenbaum 2019). While our results do not directly support this hypothesis, they pinpoint introgressed alleles in immune-related genes that could have and continue to modulate human phenotypes. Although we identified a number of likely causal NIMs in fine mapping, our strategy likely only picks up a small fraction of the functional NIMs suggesting that additional NIMs that are causal for specific traits remain to be discovered. 

Our study has several limitations due to the current availability of data and statistical methods. First, all of our analyses focus on the white British individuals in the UKBB due to the large sample size that permits the interrogation of low-frequency NIMs and our choice of NIMs based on introgressed mutations segregating in European populations. Whole-genome sequencing data in diverse populations can potentially elucidate the impact of Neanderthal introgression in other out-of-African populations that harbor substantial Neanderthal ancestry. Alternatively, designing arrays that have SNPs informative of archaic ancestry followed by genotype imputation could be a fruitful strategy to leverage large Biobanks to systematically explore the contribution of archaic introgression. Second, while our approach to localize credible NIMs yields a list of NIMs that are highly likely to modulate variation in a trait, our method only identifies a subset of causal variants. The design of fine mapping methods to study introgressed mutations while taking into account the ancestry (as well as better incorporating other measures such as posterior inclusion probabilities) is an important direction for future work. More broadly, the unique evolutionary history of introgressed variants motivate the development of methods tailored to their population genetic properties. While our results suggest potential evolutionary models that explain our observations of depleted heritability at introgressed alleles, evolutionary models that can comprehensively explain our observations are lacking. A major challenge is the large space of potential models that need to be explored. Nevertheless, proposing and validating such models will be an important direction for future work. 
                         % etc.
\chapter{Impact of Human specific variants on modern human biology}
\section{FDs}
\chapter{Evolutionary modeling of the differential contribution of Neanderthal ancestry to complex traits provides insights into selective forces that shape trait variation}
\section{Introduction}
We recently developed a methodology to assess whether Neanderthal ancestry is over- or under-represented in the genetic component of complex phenotypes compared to random genetic variation. Based on 500,000 individuals from the UK Biobank, we found the estimated contribution of Neanderthal alleles (NIMs) to phenotypic variation (NIM heritability) is significantly depleted in the great majority of the phenotypes. This is consistent with the observation that in general, natural selection has acted to remove Neanderthal alleles since introgression. On the other hand, we have found that Neanderthal alleles were significantly over-represented in their contribution to a handful of traits.
To understand the evolutionary models that could explain these observations, we performed forward-in-time population genetic simulations to model the evolution of Neanderthal and non-Neanderthal alleles according to a demographic model relating modern humans and Neanderthals. We chose parameters used in a previous study (Petr PNAS 2019) analyzing the fitness cost of Neanderthal introgression. Specifically, an ancestral population of size 10,000 diploid individuals splits into a human population and a Neanderthal population, each one evolves separately before a single pulse of Neanderthal admixture followed by subsequent random mating. Under this demography, we modeled evolution of phenotypes subject to different forces including directional, stabilizing, and disruptive selection. We estimated a NIM heritability Z-score, a measure of whether NIM heritability deviates significantly from the background alleles. We found under most models of selection, the NIM heritability Z-score is near zero or negative, indicating NIM heritability is neutral or depleted. Interestingly, we were able to recreate a positive NIM heritability Z-score, indicating an elevated Neanderthal contribution to heritability in two separate models of stabilizing and directional selection. In the stabilizing selection model, the optimal value of the trait is decreased in the human branch during the split between humans and Neanderthals leading to a positive NIM heritability Z-score. We also observe a positive NIM heritability Z-score in a directional selection model in which the parameter that couples SNP effect size and fitness is reduced after introgression. This observation highlights possible mechanisms for how complex traits evolved in human history by examining the genetic contribution of Neanderthal ancestry.

\input {chapter6}
\input {chapter7}
\input {chapter8}
\bibliography {bib/network,bib/naming}    % bibliography references
\bibliographystyle {uclathes}
\end {document}
\end{verbatim}
}
\caption {Top-Level \LaTeX\ Input File ({\tt diss.tex})}
\label {fig:toplevel}
\end {figure}
\begin {figure}
{\small
\begin{verbatim}
%%%%%%%%%%%%%%%%%%%%%%%%%%%%%%%%%%%%%%%%%%%%%%%%%%%%%%%%%%%%%%%%%%%%%%%%
%                                                                      %
%                          PRELIMINARY PAGES                           %
%                                                                      %
%%%%%%%%%%%%%%%%%%%%%%%%%%%%%%%%%%%%%%%%%%%%%%%%%%%%%%%%%%%%%%%%%%%%%%%%

\title          {Improving the Throughput \\
                of Connectionless Datagram Protocols \\
                over Networks with Limited Bandwidth}
\author         {Richard Bert Wales}
\department     {Computer Science}

%%%%%%%%%%%%%%%%%%%%%%%%%%%%%%%%%%%%%%%%%%%%%%%%%%%%%%%%%%%%%%%%%%%%%%%%

\chair          {Jack W.\ Carlyle}
\member         {Mario Gerla}
\member         {David G.\ Cantor}
\member         {Richard L.\ Baker}
\member         {Robert M.\ Stevenson}

%%%%%%%%%%%%%%%%%%%%%%%%%%%%%%%%%%%%%%%%%%%%%%%%%%%%%%%%%%%%%%%%%%%%%%%%

\dedication     {\sl To my mother \ldots \\
                who---among so many other things--- \\
                saw to it that I learned to touch-type \\
                while I was still in elementary school}

%%%%%%%%%%%%%%%%%%%%%%%%%%%%%%%%%%%%%%%%%%%%%%%%%%%%%%%%%%%%%%%%%%%%%%%%

\acknowledgments {(Acknowledgments omitted for brevity)}

%%%%%%%%%%%%%%%%%%%%%%%%%%%%%%%%%%%%%%%%%%%%%%%%%%%%%%%%%%%%%%%%%%%%%%%%

% UCLA security standards no longer allow specifying the year or place
% of birth. The degree for which this manuscript is written must also
% not be included.

\vitaitem   {1974--1975}
                {Campus computer center ``User Services'' programmer and
                consultant, Stanford Center for Information Processing,
                Stanford University, Stanford, California.}
\end{verbatim}
}
\caption {Preliminary Page Info ({\tt prelim.tex})---Part 1 of 2}
\label {fig:prelim1}
\end {figure}

\begin {figure}
{\small
\begin{verbatim}
\vitaitem   {1974--1975}
                {Programmer, Housing Office, Stanford University.
                Designed a major software system for assigning
                students to on-campus housing.
                With some later improvements, it is still in use.}
\vitaitem   {1975}
                {B.S.~(Mathematics) and A.B.~(Music),
                Stanford University.}
\vitaitem   {1977}
                {M.A.~(Music), UCLA, Los Angeles, California.}
\vitaitem   {1977--1979}
                {Teaching Assistant, Computer Science Department, UCLA.
                Taught sections of Engineering 10 (beginning computer
                programming course) under direction of Professor Leon
                Levine.
                During summer 1979, taught a beginning programming
                course as part of the Freshman Summer Program.}
\vitaitem   {1979}
                {M.S.~(Computer Science), UCLA.}
\vitaitem   {1979--1980}
                {Teaching Assistant, Computer Science Department, UCLA.}
\vitaitem   {1980--1981}
                {Research Assistant, Computer Science Department, UCLA.}
\vitaitem   {1981--present}
                {Programmer/Analyst, Computer Science Department, UCLA.}

%%%%%%%%%%%%%%%%%%%%%%%%%%%%%%%%%%%%%%%%%%%%%%%%%%%%%%%%%%%%%%%%%%%%%%%%

\publication    {{\sl MADHOUS Reference Manual.}
                Stanford University, Dean of Student Affairs
                (Residential Education Division), 1978.
                Technical documentation for the MADHOUS
                software system used to assign students to
                on-campus housing.}

%%%%%%%%%%%%%%%%%%%%%%%%%%%%%%%%%%%%%%%%%%%%%%%%%%%%%%%%%%%%%%%%%%%%%%%%

\abstract       {(Abstract omitted for brevity)}

%%%%%%%%%%%%%%%%%%%%%%%%%%%%%%%%%%%%%%%%%%%%%%%%%%%%%%%%%%%%%%%%%%%%%%%%
\end{verbatim}
}

\caption {Preliminary Page Info ({\tt prelim.tex})---Part 2 of 2}
\label {fig:prelim2}
\end {figure}

\section {Commands for Preliminary Pages}

In order to ensure that the preliminary pages
of the thesis are in the proper format,
the \verb+uclathes+ document style macros
include all the instructions necessary
to produce these pages.
All you, the student, need to do
is to specify the various pieces of information
(names, titles, etc.)
in a series of special commands as described below.
These commands should be placed at the very beginning
of the \LaTeX\ input---%
{\em before\/} the \verb+\begin {document}+ command.
The very first command
after the \verb+\begin {document}+ command
should be the \verb+\makeintropages+ command already described.

Although it is by no means mandatory,
it is strongly recommended
that you put all of the following ``declaration'' commands
pertaining to the preliminary pages
into a separate source file
(possibly with a file name like \verb+prelim.tex+
or \verb+chapter0.tex+),
and then refer to this file via an \verb+\input+ command
in your main source file.

Note that those headings on preliminary pages
which are shown in FULL CAPITALS in \regs\
have in the past appeared in {\sc Upper/Lower Small Capitals}
by the \verb+uclathes+ document style.
Also, the thesis title---%
as well as the student's name on the title page---%
appeared in large, bold type.
As of September 2016, this is no longer acceptable.
The headings of preliminary pages cannot use small caps.
The title and author must be the same size as the
surrounding text, with ``University of California" and
``Abstract of the Thesis/Dissertation" appearing in 
FULL CAPITALS and \textbf{only} the title may be bolded
if the committee allows for it. Author name cannot be
in bold font.
\footnote {See the section
``If the Dissertation Advisor Gives You Trouble''
before you go to file your manuscript.}

\subsection {Title Page Information}

The title of your thesis should be specified
via a \verb+\title+ command, as follows:

\begin {center}
\verb+\title {+{\sl text\/}\verb+}+
\end {center}

The title will be printed in normal (12-point) type,
in a \LaTeX\ \verb+center+ environment.
It is recommended that you specify explicit line breaks
via \LaTeX\ \verb+\\+ commands. The title \textit{may}
be bolded only if the committee allows, and this change
must be added manually.

Your own name should be specified
via an \verb+\author+ command, as follows:

\begin {center}
\verb+\author {+{\sl name\/}\verb+}+
\end {center}

Be sure that your name is specified {\em exactly\/}
as it appears on official University records;
otherwise, you will run into major trouble
when you attempt to file\footnote{
	Rich Wales isn't kidding.
	Another person on the Ficus project's thesis
	  was (temporarily) rejected because
	  he didn't spell out his middle name to match
	  university records.
	}.

The department name should be specified as follows:

\begin {center}
\verb+\department {Computer Science}+
\end {center}

Normally, the year in which your degree will be granted
will be the same as the current year
(that is, the year in which you are printing your manuscript).
However, if you do your printing in December
after the filing deadline for Fall Quarter,
your degree will not actually be granted until Winter Quarter,
and so you will need to specify the upcoming year
via a \verb+\degreeyear+ command, as follows:

\begin {center}
\verb+\degreeyear {+{\sl year\/}\verb+}+
\end {center}

Note that the text below the thesis title will be split across lines
in a manner slightly different from that shown in the sample title
pages in the \regs.
The text produced by the \verb+uclathes+ style will read as follows
(for a Ph.D. dissertation; similarly for a master's thesis):

\begin {center}
A dissertation submitted in partial satisfaction \\
of the requirements for the degree \\
Doctor of Philosophy in Computer Science
\end {center}

The reason I did this was to make the awkward wording specified by
the university (particularly the omission of the word ``of'' after the
word ``degree'') somewhat more palatable.
The last time I checked, the \tdadvisor\ had no objections to
this change in line formatting, as long as the wording itself were
not changed.%
\footnote {The \tdadvisor\ will {\em not\/} accept a manuscript
in which the word ``of'' appears after ``degree'' on the title page.
See the section below, ``Possible Future Developments'',
for more on this issue.}

\subsection {Copyright}

In most cases,
you will not need to include any special commands at all
for the copyright page.
It will be generated automatically, using your name
and the year in which the degree is to be granted.
The following commands exist to cover unusual situations.

\begin {itemize}

\item
In the unlikely event that you have already
published your thesis prior to filing,
you should specify the actual year of copyright
(the year of first publication) as follows:

\begin {center}
\verb+\copyrightyear {+{\sl year\/}\verb+}+
\end {center}

If this year is different from the year
in which the degree is granted,
\LaTeX\ will include both years (separated by a comma)
in the copyright notice.

It is not necessary
to include a \verb+\copyrightyear+ command
solely because you are filing in December
and will not be receiving your degree until Winter Quarter.

\item
If, for some unusual reason,
you explicitly do not wish to include
a copyright notice in your manuscript,
you can suppress it via the following command:

\begin {center} \verb+\nocopyright+ \end {center}

Note that, under the provisions of the Berne copyright convention,
which went into effect on 1~April 1989, and to which the U.S.\ is a
signatory, your thesis is considered to be copyrighted
even if you omit the copyright notice.

\end {itemize}

The \regs\ currently do not permit the inclusion of the
phrase ``All Rights Reserved'' in the copyright notice of a
thesis or dissertation.%
\footnote {This phrase is essential for proper copyright protection
in certain South American countries which are signatories {\em only\/}
to the older Pan American Copyright Convention.
Although ``All Rights Reserved'' has no extra effect
under U.S.\ copyright law, virtually all books currently published
in the United States include this phrase in the copyright notice.}
See the section below, ``Possible Future Developments'',
for more on this issue.


\subsection {Signature Page}

The members of your thesis committee are specified via
\verb+\chair+ and \verb+\member+ commands.
Use a separate command for each committee member.

The committee chair is specified as follows:

\begin {center} \verb+\chair {+{\sl name\/}\verb+}+ \end {center}

Each remaining committee member is specified as follows:

\begin {center} \verb+\member {+{\sl name\/}\verb+}+ \end {center}

If the first two members of your committee are co-chairs, then use two
\verb+\chair+ commands to list them.

The members of the committee should be named in the \emph{same order}
as they appeared on your ``Nomination of Committee'' form
(with the chair or co-chairs first).
\LaTeX\ will output the names in the {\em opposite\/} order
(as required by \ucla\ policy)
when it generates the signature page.

Also, be sure that each committee member
is identified using his or her {\em full name}---%
including middle initial, if any---%
as specified in the General Catalog.

\subsection {Dedication and Acknowledgments}

If you wish to include a {\em dedication\/} in your manuscript,
specify it via the following command:

\begin {center} \verb+\dedication {+{\sl text\/}\verb+}+ \end {center}

The {\sl text} will be formatted by \LaTeX\
in a \verb+center+ environment,
centered vertically on a page by itself,
and in regular type.
If you wish to use {\it italics\/} (\verb+\it+)
or {\sl slanted\/} (\verb+\sl+) type
for the dedication, you must specify this yourself.

If you wish to include {\em acknowledgments,}
specify them via the following command:

\begin {center}
\verb+\acknowledgments {+{\sl text\/}\verb+}+
\end {center}

The {\sl text} will be formatted by \LaTeX\ using a regular environment
(normal, justified margins).

Some people get confused as to what should be in a ``dedication'' and
what should be in ``acknowledgments''.
Here are some guidelines which should help:

\begin {itemize}

\item
A {\em dedication\/} is a way of making particular mention
of a person who is very close and special to you---%
such as a spouse, other family member, very close friend,
or other person who has been particularly instrumental
or supportive in your life,
and without whose influence the degree work
(or even your entire academic career)
might never have been completed.

A dedication should be used only
when there is a close emotional bond
between you and the person named.
Unless the dedication is expected to
have a deep emotional effect both on you and on the other person,
you should seriously consider instead
mentioning him or her in your ``Acknowledgments'',
if at all.

A proper dedication will almost always
start out with the word ``to'',
and will rarely if ever be a complete sentence
(though it is perfectly proper to include
a brief explanation of why the person named
in the dedication has been important to you).
A dedication should not normally be
more than three or four lines long.

\item
{\em Acknowledgments\/} are a way of thanking people
who were helpful and/or supportive in your work---%
such as professors or employers,
as well as family members who have been helpful and understanding.
This is also the place to mention instances
in which you have obtained permission from others
for the use of their copyrighted material in your thesis
(see \regs\ for more detail on this subject).

Acknowledgments should always
be in the form of complete sentences.

\end {itemize}


\subsection {Vita, Publications, and Presentations}

Vita (life history), publication, and presentation information
should be included only in doctoral dissertations---%
not in master's theses.

Each ``vita'' item should be specified
via a separate command of the following form:

\begin {center}
\verb+\vitaitem {+{\sl date\/}\verb+} {+{\sl text\/}\verb+}+
\end {center}

Each ``vita'' item is printed in two columns:
a narrow column on the left for the {\sl date,}
and a wider column on the right for the {\sl text}.

The ``vita'' items should be specified in chronological order,
as they will be printed in the order given.

Each ``publication'' or ``presentation'' item
should be specified via a separate command
of one of the following forms:

\begin {center}
\verb+\publication {+{\sl text\/}\verb+}+ \\
\verb+\presentation {+{\sl text\/}\verb+}+
\end {center}

Each ``publication'' and/or ``presentation'' item
is printed as free-form text.
If you wish the items to appear in any particular
bibliography-like format,
it is your responsibility to do all necessary formatting yourself.

The ``publication'' and/or ``presentation'' items
will be printed in a single, unified list,
in the order specified.
The \verb+uclathes+ document style macros
will automatically generate the proper heading
as appropriate.


\subsection {Abstract}

The text of the abstract should be specified
via a command of the following form:

\begin {center}
\verb+\abstract {+{\sl text\/}\verb+}+
\end {center}

All of the ``heading'' information on the abstract page
is taken from the corresponding data
for the title and signature pages,
and need not be specified a second time.



\section {Things to Avoid}

In order to ensure that your thesis manuscript
will conform to University requirements,
it is essential that you {\em do not\/} indulge in
certain practices that would disrupt the standard format.
The following list is not intended to be exhaustive,
but indicates various common things
which you must not do:

\begin {itemize}

\item
Do not attempt to change the margins.
Most \verb+\setlength+ commands affecting such values as
\verb+\textwidth+, \verb+\textheight+, \verb+\topmargin+,
\verb+\oddsidemargin+, or \verb+\evensidemargin+
will render your thesis unacceptable to the \tdadvisor.


\item
Do not disturb the page numbering.
In particular, you must not attempt
to use a page-numbering scheme
in which the page number starts over for each new chapter
(for example, page number ``3-1'' for the first page of Chapter~3).

The University regulations are quite strict
regarding the required method of numbering pages;
any deviation from the default scheme
will result in an unacceptable manuscript.

\item
Do not try to specify small type sizes.
The default (12-point) type used by the \verb+uclathes+ style macros
is the smallest acceptable size for a thesis manuscript.%
\footnote {The occasional (default) appearance of smaller type
in superscripts, footnotes, and the like is acceptable;
don't worry about this.}

\item
Do not try to use marginal notes
(\verb+\marginpar+ command).
Absolutely no text---%
not even occasional marginal notes---%
is permitted to fall outside the official margins.

\end {itemize}

Additionally, the following practices---%
while not explicitly illegal---%
are either unlikely to give pleasing results
or are liable to create serious problems,
and should therefore be avoided or used with great care:

\begin {itemize}

\item
{\em Double-column text\/}
is not explicitly prohibited by the University guidelines.
However, given the required type size
for the body of the manuscript,
double-column text will probably detract considerably from readability,
and it should therefore not be attempted.

\item
{\em Running footers\/} are permissible.
However, you must take special care to ensure
that the footer text remains within the margins for regular text
(that is, at least 1.25 inches from the bottom edge of the paper),
and that the page number remains in its default position.
In particular, the page number may not be moved up
onto the same line with a running footer.

\end {itemize}

\section {Line Spacing in Manuscripts}

Traditionally, \ucla\ has required that the text of a thesis
be double-spaced (3~lines per inch).
While this requirement is generally considered
to be appropriate for typewritten material,
many people feel that double-spacing
of the output of modern laser printers
is not necessary to ensure easy readability of the manuscript,
and may indeed detract from readability.
\newpage
Even though \regs\ requires double spacing,
it appears that the \ucla\ \tdadvisor\ will in fact accept
manuscripts with one-and-a-half spacing
(that is, 4.5 lines per inch)
under certain conditions.
In order to produce one-and-a-half spacing,
add a comma and the word \verb+single+
to the \verb+\documentclass+ line, thusly:

\begin {center}
\verb+\documentclass [PhD,single] {uclathes}+ \\
{\em or} \\
\verb+\documentclass [MS,single] {uclathes}+
\end {center}

The \verb+single+ option will not affect the spacing
of the {\em abstract.}
The \tdadvisor\ continues to require abstracts to be double-spaced.%
\footnote {One good reason for this is that University Microfilms
International (UMI), the company which microfilms dissertations,
transcribes the abstract of each dissertation into their database
system.
UMI has stated that it finds double-spaced abstracts
much easier for their personnel to transcribe.}

If you intend to use the \verb+single+ option,
it is imperative that you bring a sample of your manuscript
to the \tdadvisor\ (1255 Murphy Hall, 310-825-3819, academicservices@grad.ucla.edu)
for review and approval
well in advance of the filing deadline---%
if at all possible, by the beginning of the quarter
during which filing will take place.
The material submitted for review
should include all preliminary pages,
as well as a representative sampling of the body of the text.
This is crucially important, for two reasons:

\begin {itemize}

\item
The last time I checked, the \oldtdadvisor\ insisted that anyone
wanting to use closer than double line spacing must bring a sample
to her office in advance of filing.
This is something she had been encouraging students to do for
years anyway (though with only limited success). \textbf{As of
2012, it is unclear if this is still encouraged. Ryan did find
the \tdadvisor\ accessible and available by email during the
filing process.}

\item
In case, for some reason, your manuscript is found unacceptable,
you will still have plenty of time to reprint it in a form that
the \tdadvisor\ will allow you to file.

\end {itemize}

\section{Formatting Your Thesis In Two Ways}

\MaintainNote{This topic is new, which explains why it's 
	not as carefully documented or implemented as Rich Wales'
	code.  My apologies, but I wanted to graduate.
	Also, the ``I'' in this section refers to me,
	not Rich.   ---John.}

One of the virtues of \LaTeX{} is that it's mostly a markup
  language---the user indicates what things are, rather than
  explicitly how they should be rendered.
I took advantage of this capability to format my thesis two different
  ways,  an official ``submission'' version and a working version for
  me, my committee, and a technical report.
I did this because the double-spaced, single column format
  \emph{required} by the thesis committee was designed for typewritten
  theses of 20 years ago rather than 
  good design rules of the professional design world.\footnote{
	What's wrong with it?  IMHO, double-spacing has nearly no place
	in the modern world (except in very drafty documents);
	double-spacing makes the document about 50\% longer than
	it needs to be and it also make it difficult to fit some
	tables on a single page.  Double-spacing is actually a saving
	grace, though, because it makes up for the fact the the
	lines are too wide to comfortably read.
	}

If you're interested in trying the ``two-outputs'' approach,
  look at the \texttt{demo\_techreport} example.


\section {Possible Future Developments}

I am planning, in the near future, to approach the Graduate Division
in order to propose three changes to the current \regs:

\begin {itemize}

\item
Make the current {\em de facto\/} acceptance
of one-and-a-half line spacing (4.5 lines per inch) official.

\item
Permit the phrase ``All Rights Reserved'' in the copyright notice,
in order to bring the notice in line with what almost all book
publishers do, as well as to give more protection to students from
those countries which require this phrase as part of a legal and
enforceable copyright notice.

\item
Add the word ``of'' after the word ``degree'' on the title page.
If I cannot get this changed, I will propose that the sample title
pages in the \regs\ be modified to put the degree name on a line
by itself (start a new line after the word ``degree'')---%
which is the way the \verb+uclathes+ style will do it already.

\end {itemize}

If any changes are adopted in these areas by the Graduate Division,
I will modify the \verb+uclathes+ macros accordingly.
In particular, if one-and-a-half line spacing becomes official,
I will probably modify the macros to make such spacing the default.


\MaintainNote{These were Rich Wales' future developments.
I do not know the status of either, and I am not pursuing them myself.
   ---John.}


\section {If the Dissertation Advisor Gives You Trouble}

As mentioned earlier, you should plan to bring a sample of your
manuscript to the Thesis and Dissertation Office well in
advance of your planned filing date.
This is especially true if you plan to use the \verb+single+
line-spacing option---%
but you should do it in any case.
The material you show the Advisor
should include {\em all\/} the preliminary pages,
as well as a representative sample of your text.

There is a possibility that the \tdadvisor\ may object to some
aspect of your manuscript.
If it appears that the Advisor is unwilling to accept some feature
that is produced via the \verb+uclathes+ macros
(for example, the selection of fonts on the preliminary pages,
or the line spacing if you are using the \verb+single+ option),
please get as much {\em specific\/} information as possible
regarding what they are objecting to,
and let us know right away.

\MaintainNote{I'm interested
  in hearing about objections the \oldtdadvisor\ has to
  the default format provided by \texttt{uclathes}.
  Since it has been accepted for my dissertation,
  it is correct as of 1995.
   ---John.}

\MaintainNote{We are interested
  in hearing about any issues you face with the
  \tdadvisor\ regarding the \texttt{uclathes} template. It has been used
  for three manuscripts that were approved as of April 2017.
  ---Ryan.}

\end {document}

% LocalWords:  umi uclathes microfilming snooze documentclass ms phd annote tex
% LocalWords:  bibliographystyle diss Connectionless Gerla San prelim Los Berne
% LocalWords:  MADHOUS signatory signatories co de markup IMHO rep
